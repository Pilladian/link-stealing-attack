\chapter{Background}

  \section{Neural Networks}

    Neural Networks (NNs) are key components in Artificial Intelligence (AI) and Deep Learning.
    They try to simulate some properties and the functionality of biological neural networks, like our brain, by imitating the way biological neural systems process data.
    Today, they have been applied successfully to speech recognition, face recognition on images or the transformation from speech to text.
    They are used to model software agents in video games, let autonomous robots learn new things or find patterns in data.

    The simplest class of Neural Networks are Feed Forward Neural Networks.
    Like other classes, they consists of multiple layers.
    An input layer, one or more hidden layers and an output layer.
    Each of these layers contain multiple neurons, which are represented as mathematical functions, that take multiple inputs and use them to calculate one output.
    Such neuron is also called perceptron, while a fully connected neural network is called a multilayer perceptron (MLP).

    %\vspace{0.48cm}
    \begin{figure}%[!h]
      \begin{center}
        \begin{tikzpicture} [node distance = 2cm,
          on grid,
          %auto,
          %every state/.style = {draw = blue, fill = blue!30},
          every initial by arrow/.style = {font = \large,
              thick,-stealth
          }
          ]
          
          % Input Layer
          \node (i1) [state, initial, initial text= $input_1$] at (0,1) {};
          \node (i2) [state, initial, initial text= $input_2$] at (0,-1){};

          % Hidden Layer 1
          \node (h11) [state] at (2,2) {};
          \node (h12) [state] at (2,0) {};
          \node (h13) [state] at (2,-2) {};

          % Hidden Layer 2
          \node (h21) [state] at (4,2) {};
          \node (h22) [state] at (4,0) {};
          \node (h23) [state] at (4,-2) {};

          % Hidden Layer 2
          \node (h31) [state] at (6,2) {};
          \node (h32) [state] at (6,0) {};
          \node (h33) [state] at (6,-2) {};
          
          % Output Layer
          \node (o1) [state, accepting right, accepting text = {$output_1$}] at (8,1) {};
          \node (o2) [state, accepting right, accepting text = {$output_2$}] at (8,-1){};

          % Links
          \path [-stealth, thick]
            (i1) edge node {}   (h11)
            (i1) edge node {}   (h12)
            (i1) edge node {}   (h13)
            (i2) edge node {}   (h11)
            (i2) edge node {}   (h12)
            (i2) edge node {}   (h13)
            
            (h11) edge node {}   (h21)
            (h11) edge node {}   (h22)
            (h11) edge node {}   (h23)
            (h12) edge node {}   (h21)
            (h12) edge node {}   (h22)
            (h12) edge node {}   (h23)
            (h13) edge node {}   (h21)
            (h13) edge node {}   (h22)
            (h13) edge node {}   (h23)
            
            (h21) edge node {}   (h31)
            (h21) edge node {}   (h32)
            (h21) edge node {}   (h33)
            (h22) edge node {}   (h31)
            (h22) edge node {}   (h32)
            (h22) edge node {}   (h33)
            (h23) edge node {}   (h31)
            (h23) edge node {}   (h32)
            (h23) edge node {}   (h33)

            (h31) edge node {}   (o1)
            (h31) edge node {}   (o2)
            (h32) edge node {}   (o1)
            (h32) edge node {}   (o2)
            (h33) edge node {}   (o1)
            (h33) edge node {}   (o2);
        
        \end{tikzpicture}
      \end{center}

      \caption{Multilayer Perceptron}
      \label{figure:neural-network}
    \end{figure}

    \subsubsection*{Train a Neural Network}

      When we are confronted to a new task, we try to gain as much information as possible and based on them, we aim to learn how to solve the given problem.
      Neural networks behave similar.
      Before we can apply a neural network on classifying whether the object we provide is a spoon or fork, we need to tell the network, based on which information it should make its decision.
      That could be images of spoons and forks or their weight, length and width.
      We call this data \emph{Training Data}.
      To train the model in our example, we provide an image of a spoon and let the network make its decision.
      If the decision is correct, we wont change anything.
      But if the decision is incorrect, we need to slightly modify the weights - the connections between the perceptrons - to let the model behave different in the future.
      We do this for each image in our training set and repeat this process $n$ times (for $n$ epochs).
      After the training phase, we hopefully have a good trained neural network, which performs well on the provided task.

	\section{Graphs}

		As Graph we denote a data structure that contains nodes and edges. 
    Let $G = (V, E)$ be a graph with $V$ being the set of nodes and $E$ being the set of edges.
    We denote $\overrightarrow{a}$ as the feature vector of node $a$, representing the attributes of $a$.
    An edge $e = (i,j)$ contains the source node $i$ and the destination node $j$.
    In that way, links describe the relationship between the source and destination node. 
    The most popular example, where graphs are used to model data, are social networks. 
    The nodes represent the users that have multiple attributes like location, gender, workplace etc., while the edges state which relationship the users have.
    Lets consider a directed graph $G = (V,E)$ with $V$ representing the users and $E$ their relationships.
    If user $a$ follows user $b$, the edge $e_{ab} = (a,b)$ would be an element in $E$.
    If user $b$ follows user $a$, the edge $e_{ba} = (b,a)$ would be an element in $E$.
    Lets consider an undirected graph $G' = (V',E')$ with $V'$ representing the users and $E'$ their relationships.
    In $G'$ it doesn't matter whether user $a'$ follows $b'$ or the other way around. 
    Both results in $e'_{a'b'}, e'_{b'a'} \in E'$.
    Figure \refeq{figure:undirected-graph} shows an undirected graph.
    Since there is a link drawn between node $A$ and node $B$, we know, that there exists some relationship between them, but we don't know "who follows whom" ($e_{AB}, e_{BA} \in E$).
    The same holds for $e_{CF}, e_{FC} \in E$ or $e_{EG}, e_{GE} \in E$.
    Since there isn't a link drawn between node $E$ and node $C$, there doesn't exist a known relationship between them ($e_{EC}, e_{CE} \not\in E$).

    \vspace{0.48cm}
    \begin{figure}[!h]
      \begin{center}
        \begin{tikzpicture} [node distance = 2cm,
          on grid,
          %auto,
          %every state/.style = {draw = orange, fill = orange!30},
          every initial by arrow/.style = {font = \large,
              thick,-stealth
          }
          ]
          
          \node (a) [state] at (-1,2) {$A$};
          \node (b) [state] at (0,-1){$B$};
          \node (c) [state] at (3,1){$C$};
          \node (d) [state] at (5,-1){$D$};
          \node (e) [state] at (1,3){$E$};
          \node (f) [state] at (3,-2){$F$};
          \node (g) [state] at (7,3){$G$};

          % Links
          \draw (a) -- (b);
          \draw (d) -- (c);
          \draw (f) -- (a);
          \draw (a) -- (e);
          \draw (b) -- (f);
          \draw (c) -- (f);
          \draw (c) -- (g);
          \draw (d) -- (g);
          \draw (a) -- (c);
          \draw (e) -- (g);
        
        \end{tikzpicture}
      \end{center}

      \caption{Undirected Graph}
      \label{figure:undirected-graph}
    \end{figure}

	\section{Graph Neural Networks}
      
    % Describe how to query a Graph Neural Network (f(Graph, node))
    In the last decades, social networks have become a huge part of life for many people all around the world.
    The companies behind these networks collect tons of data of each user every day.
    Let $G = (V, E)$ be a graph that models such a social network. 
    The set of all users is given as $V$ and $E$ represents their relationships.
    The feature vector $\overrightarrow{a}$ of user $a$ contains all information the company already has regarding this user.
    That could be the name, relationship status, workplace, address, amount of children and so on.
    Lets consider one attribute (\emph{workplace}) as non mandatory for the registration.
    This leads to some users $v_{workplace} = \{v $ | $ \forall v \in V: v$'s $workplace$ is known$\}$, the company knows the workplace from and some users $v_{unknown} = \{v $ | $ \forall v \in V: v$'s $workplace$ is unknown$\}$, where the attribute is unknown.
    Based on the information $G$ provides, a Graph Neural Network model $f$ can be trained to predict the missing attribute of all users in $v_{unknown}$.
    This is a simple node classification task, where the missing attribute $workplace$ is the label.
    This example is one of many tasks a Graph Neural Network can perform.
    Others could be graph classification, where $f$ should be able to predict whether a given graph is a subgraph of another one, or link prediction, which is used for friendship prediction in social networks.
    There exist two major ideas of training a Graph Neural Network - transductive or inductive.
    In our experiments introduced and described in Chapter \refeq{chapter:attacks} we focus on attacks on inductive trained Graph Neural Networks. 

    \subsection*{Transductive Setting}

      In the transductive setting \cite{kipf2017semisupervised} we consider the graph being fixed.
      Meaning, that neither the links nor the feature vectors of the nodes change over the time.
      Furthermore, the complete graph is provided as input to a Graph Neural Network model $f$, including all nodes, their links and their feature vectors.
      Thus, the model learns hidden layer representations that encode both local graph structure and features of nodes to perform its task, by aggregating information throughout the graph.
      In the example given above, we want to predict the $workplace$ of unlabeled users.
      Since this setting operates on the full graph, we consider the information the graph provides as known all the time, especially during the training phase.
      This implies, that all feature vectors are visible for $f$ during training.
      Meaning, that if new users join the network, $f$ needs to be retrained because the graph changed, making it really hard to apply this setting to real world problems.
      Datasets like social networks change every day, such that a model which is trained transductive, must be retrained as often as the structure changes, to maintain their accuracy.
      This leads to high computational costs and time constraints and thus is not very practical for very large, constant changing datasets.
      Based on this problem another learning method can be used, which does not use the full graph for training but uses partial graphs instead.

    \subsection*{Inductive Setting}

      For inductive learning, we do not provide the full graph as input for the Graph Neural Network model $f$.
      Instead an aggregation function $aggregate(n)$ is used, which aggregates the feature vectors of the $k$-hop neighborhood of the node $n$ to obtain the input for $f$. 
      We can set $k$ to any number.
      $k = 0$ does only consider $n$'s feature vector as input and no aggregated neighborhood embeddings.
      $k = 1$ aggregates the feature vectors of $n$'s neighbors with $n$'s own feature vector, while $k = 2$ also considers the neighbors of the neighbors of $n$.
      Besides this aggregation function the model also learns an update function $update(n)$, which updates $n$'s feature vector with the aggregation of it's neighborhood.
      In that way $\overrightarrow{n}$ not only contains $n$'s features but also the information of it's neighbors.
      Thanks to this algorithms, it's no longer necessary to train $f$ on the whole graph.
      To better understand the process, we continue with the example given above.
      Lets assume that three new users register to the social network, nobody providing their workplace.
      Since some other mandatory information and maybe some links are given, the Graph Neural Network can be used to predict the workplace of the new users nevertheless they haven't been included in the training process.
      This happens, by aggregating their neighborhood feature vectors with the own one and querying $f$ on the updated feature vector.
      $f$ will then predict the workplace based on the provided aggregation, since it was trained to find patterns in vectors that lead to certain predictions.
      Therefore, inductive trained Graph Neural Networks are able to generalize to unseen nodes, making them easier to use for real world problems.
      
      Since the most real world problems like friendship prediction in social networks or protein-protein interactions in chemical networks cannot be modeled with a static graph, the inductive learning method is more widely used compared to the transductive learning method.
      
  \section{Link Stealing Attacks}
    
    Link Stealing Attacks have been introduced the first time by He et al. \cite{DBLP:journals/corr/abs-2005-02131}.
    Given a Graph Neural Network, they performed different attacks with different attack methodologies to steal links from the graph that was used for training the target GNN.

    Let $f$ be the target Graph Neural Network model, that was trained on some graph dataset $G = (V, E)$.
    We assume the adversary was able to obtain some partial graph of $G$ and define it as $G_s = (V_s, E_s)$.
    Furthermore, we assume that $E_s$ is incomplete.
    This means, that some links that originally have been in $E$ are missing in $E_s$.
    The goal of link stealing attacks is to recover these missing edges.
    To achieve the reconstruction, the target model will be queried on two nodes $i,j \in V_s$.
    Since $f$ was trained to perform some task, the outputs $f(i)$ and $f(j)$ will be some posterior representing $f$'s prediction.
    Based on those posteriors, an adversary will train an attack model $A$, that infers, whether the two nodes $i$ and $j$ have originally been connected in $G$, and therefore the edge $e_{ij}$ is missing in $E_s$, or if they haven't been linked in the first place.
    To train the attack model, one need some positive and some negative samples.
    We define the positive samples $pos = \{ (i,j)$ | $i,j \in V_s: e_{ij} \in E \wedge e_{ij} \not\in E_s\}$ as set of node pairs, that have been connected in $G$ but aren't connected in $G_s$. 
    The negative samples $neg = \{ (i,j)$ | $i,j \in V_s: e_{ij} \not\in E \wedge e_{ij} \not\in E_s\}$ are defined as set of node pairs, that haven't been connected in neither of the two graphs.
    As input for $A$ the authors used the concatenation of the two posteriors or a vector containing the results of eight common distance functions, to describe the similarity of the two posteriors $f(i)$ and $f(j)$.
    The basic intuition is the following: 
    The posterior outputs $f(i)$ and $f(j)$ should be more similar, if the two nodes $i$ and $j$ have been connected in $G$.
    If they haven't been connected in $G$, then $f$ would output two posteriors, that are less similar.

    By completing $G_s$, an adversary steals sensitive information about the training graph of the target model $f$, since $G_s$ is a partial graph of $f$'s training graph $G$.

    He et al. performed their attacks on transductive trained Graph Neural Networks and achieved high accuracy in recovering the links of $G_s$. 
    While the authors used GNNs that only can handle fixed graphs, 
    in our work, we focused on link stealing attacks on inductive trained Graph Neural Networks.
    Using the inductive training setting, our models are able to generalize to unseen nodes, leading to different model predictions based on different information the graph provides. By the end of this thesis however, we will show, that inductive trained GNNs are very likely to reveal sensitive information about their training graph as well.