\chapter{Introduction}

	% Figure Example
	% \begin{figure}
		% 	\lstinputlisting[language=C, firstline=\interestingstart, lastline=\interestingend]{\somecfile}
		% 	\caption{caption}
		% 	\label{code:aes_unsealdata}
		% \end{figure}

	\section{Motivation}
		A graph is a datastructure which is used to model large data and the relationships between entities.
		It consists of nodes and edges and can be used to model data in almost every domain.
		For example in social networks, healthcare analytics or protein-protein interactions.
		In a social network, the nodes would be the users that are registered and the edges would represent whether the users know each other or not by connecting them or not.
		A graph itself can be deemed as intellectual property of the data owner, since she may spent lots of time and resources collecting and preparing the data.
		In most cases the graph is also highly confidential because it contains sensitive information like private social relationships between users in a social network or medical information about specific people in healthcare-analytic datasets.
		Since nowadays graphs are a common way to store and visualize data, Machine Learning algorithms have been improved to directly operate on them.
		These Machine Learning Models are called Graph Neural Networks.
		



	\section{Outline}
		\TODO{write at the end}

Some citation\cite{Anderson:1972}
